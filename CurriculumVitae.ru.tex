\documentclass[11pt,a4paper]{moderncv}
\usepackage[scale=0.8]{geometry}

\usepackage[T2A]{fontenc}
\usepackage[utf8]{inputenc}

\usepackage[english,russian]{babel}


\usepackage{graphicx}

\usepackage{cite}

\usepackage{epstopdf}

\uchyph=0

\renewcommand{\rmdefault}{cmr}
\renewcommand{\sfdefault}{cmss}
\renewcommand{\ttdefault}{cmtt}

\makeatletter
\newcommand{\rmnum}[1]{\romannumeral #1}
\newcommand{\Rmnum}[1]{\expandafter\@slowromancap\romannumeral #1@}
\makeatother

\moderncvtheme[green]{casual}



\firstname{Илья}
\familyname{Меньшиков}
\mobile{+7~(922)~146~65~28}
\email{unkmas@gmail.com}
\quote{unkmas}

\makeatletter

\begin{document}
\maketitle

\section{Обо мне}
\cvitem{Интро} {
  Привет! Меня зовут Меньшиков Илья, я веб-разработчик.
}
\cvitem{} {
  Последние 9 лет я занимаюсь веб-разработкой, в основном на Ruby. Предпочитаю заниматься бэкэндом, хоть и умею работать в флустеке. Кроме Ruby немного писал в прод на Elixir, Python, Go, Js, PHP и даже C\#.
}
\cvitem{} {
  Своей работой считаю решение проблем бизнеса и пользователей, а не бездумное закрытие задач в джире. Стараюсь находить баланс между продуктовыми задачами и техническими потребностями проекта.
}
\cvitem{} {
  Успел поработать в маленьких и огромных компаниях, в продуктовой и в заказной разработке, в стартапе без денег, в стартапе с деньгами, в старом и развитом продукте. "Потрогал"\ почти все аспекты работы над продуктом - бэкэнд и фронтэнд, работа над инфраструктурой, управление проектом,  постановка задач, продуктовый анализ, менторство и ведение команды.
}
\cvitem{} {
  В прошлом я занимался научной работой в области компьютерной лингвистики. Темами моих исследований были: анализ тональности текста и подходы к автоматическому построению тезаурусов.
}

\cvitem{Цели в жизни} {
  Вырасти в топового специалиста.
}

\cvitem{Хобби}{
  Я люблю музыку, программирование, науку и кофе. Играю в Dark Souls.
}

\section{Образование}

\cventry{2015---2017}
  {Аспирант (неокончено), 09.06.01}
  {Институт Фундаментального Образования, УрФУ}
  {Екатеринбург, Россия}
{}{}
\cventry{2013---2015}
  {Магистр информационных технологий}
  {Институт Фундаментального Образования, УрФУ}
  {Екатеринбург, Россия}
{}{}
\cventry{2009---2013}
  {Бакалавр информационных технологий}
  {Физико - Технологический Институт, УрФУ}
  {Екатеринбург, Россия}
{}{}

\section{Опыт работы}

\cventry{2016---2018}
  {Разработчик}{}
  {Worki (MRG)}
  {}
{
Разработка бэкэнда на Ruby и Elixir. Решение инфраструктурных задач, проектирование архитектуры приложения. Участие в проработки фич, организация интеграций.
}
\clearpage

\cventry{2016---2018}
  {Разработчик}{}
  {СКБ-Контур}
  {}
{
Разработка бэкэнда и фронтэнда, проектирование решений для заказчиков.
\begin{itemize}
\item Разработка системы кадрового документооборота для компании "Декатлон";
\item Разработка портала банковских гарантий;
\end{itemize}
}

\cventry{2014---2016}
  {Разработчик}{}
  {Проект под NDA}
  {}
{
Ведение всей технической части проекта - руководство разработкой, разработка бэкэнда и фронтэнда, проработка и автоматизация инфраструктуры. Частично - проджект-менеджмент, постановка задач.
}

\cventry{2011---2014}
  {Разработчик}{}
  {Издательский Дом АБАК-ПРЕСС}
  {Екатеринбург, Россия}
{
\begin{itemize}
  \item Разработка портала \href{http://pulscen.ru}{PULSCEN.ru} на платформе Ruby on Rails;
  \item Развитие системы денормализации на основе PGQ;
  \item Разработка портала \href{http://dkvartal.ru}{DKVARTAL.ru} на платформе Ruby on Rails;
  \item Заладывание основ функционального и юнит-тестирования на проекте {DKVARTAL.ru};
  \item Создание вики-системы и ряда nlp-алгоритмов на сайте {DKVARTAL.ru}.;
  \item Создание ряда библиотек на языке ruby (Sitemap, EasyMailChimp, Pochatok)
\end{itemize}
}

\cvitem {}{}

\cventry{2011---2011}
  {Разработчик}{}
  {Элитмастер}
  {Екатеринбург, Россия}
{
\begin{itemize}
  \item Разработка CMS, основанной на фрэймворке Yii;
  \item Разработка корпоративной системы {1litr};
  \item Разработка и поддержка большого числа веб-сайтов
\end{itemize}
}


\section{Языки}
\cvlanguage{Русский}{Родной}{}
\cvlanguage{English}{Intermediate}{}

\section{Навыки}
\subsection{Языки и фреймворки}
\cvcomputer{Ruby}{Ruby~on~Rails, Sinatra}
{JavaScript}{JQuery, Vue}
\cvcomputer
{Имел опыт работы} {Elixir, Go, Python}
{}{}
\subsection{Окружение}
\cvcomputer{Базы Данных}{PostgreSQL, Redis}
           {Поиск} {Elasticsearch}
\cvcomputer{Прочее}{Linux, Docker, Git}
           {}{}

\section{Социальные сети}
\cvitem{GitHub}{\href{https://github.com/unkmas}{github.com/unkmas}}
\cvitem{VK}{\href{http://vk.com/unkmas}{vk.com/unkmas}}{}{}

\nocite{*}
\bibliographystyle{unsrt}
%\bibliography{unkmas}

\end{document}
