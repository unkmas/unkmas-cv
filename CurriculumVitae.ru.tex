\documentclass[11pt,a4paper]{moderncv}
\usepackage[scale=0.8]{geometry}

\usepackage[T2A]{fontenc}
\usepackage[utf8]{inputenc}

\usepackage[english,russian]{babel}


\usepackage{graphicx}

\usepackage{cite}

\usepackage{epstopdf}

\uchyph=0

\renewcommand{\rmdefault}{cmr}
\renewcommand{\sfdefault}{cmss}
\renewcommand{\ttdefault}{cmtt}

\makeatletter
\newcommand{\rmnum}[1]{\romannumeral #1}
\newcommand{\Rmnum}[1]{\expandafter\@slowromancap\romannumeral #1@}
\makeatother

\moderncvtheme[green]{casual}



\firstname{Илья}
\familyname{Меньшиков}
\mobile{+7~(922)~146~65~28}
\email{unkmas@gmail.com}
\quote{unkmas}

\makeatletter

\begin{document}
\maketitle

\section{Обо мне}
\cvitem{Представление} {
  Тут могло бы быть подробное описание
}

\cvitem{Цели в жизни} {
  Вырасти в топового специалиста; оставить свой след в науке.
}
\cvitem{Хобби}{
  Я люблю музыку, программирование, науку и кофе. Играю в настолки и увлекаюсь иллюзиями.
}
\cvitem{Исследования}{
  Я делаю свои первые шаги в науке, и мне это чертовски нравится. В данный момент, я заинтересован в NLP/CL системах и алгоритмах, в частности, в автоматическом извлечении данных из текстов на естественном языке.
}

\section{Образование}


\cventry{2015---}
  {Аспирант, 09.06.01}
  {Институт Фундаментального Образования, УРФУ}
  {Екатеринбург, Россия}
{}{}
\cventry{2013---2015}
  {Магистр информационных технологий}
  {Институт Фундаментального Образования, УРФУ}
  {Екатеринбург, Россия}
{}{}
\cventry{2009---2013}
  {Бакалавр информационных технологий}
  {Физико - Технологический Институт, УРФУ}
  {Екатеринбург, Россия}
{}{}

\section{Опыт работы}

\cventry{Март~2014---}
  {Разработчик}{}
  {Проект под NDA}
  {}
{
\begin{itemize}
  \item Ведение бэкэнда на Ruby;
  \item Ведение и автоматизация инфраструктуры;
  \item Частично - ведение фронтенда;
  \item Частично - ведение проекта, постановка задач;
\end{itemize}
}

\cventry{Октябрь~2011---Март~2014}
  {Разработчик}{Отдел веб-разработок}
  {Издательский Дом АБАК-ПРЕСС}
  {Екатеринбург, Россия}
{
\begin{itemize}
  \item Разработка портала \href{http://pulscen.ru}{PULSCEN.ru} на платформе Ruby on Rails;
  \item Развитие системы денормализации на основе PGQ;
  \item Разработка портала \href{http://dkvartal.ru}{DKVARTAL.ru} на платформе Ruby on Rails;
  \item Заладывание основ функционального и юнит-тестирования на проекте {DKVARTAL.ru};
  \item Создание вики-системы и ряда nlp-алгоритмов на сайте {DKVARTAL.ru}.;
  \item Создание ряда библиотек на языке ruby (Sitemap, EasyMailChimp, Pochatok)
\end{itemize}
}

\cvitem {}{}

\cventry{Август~2011---Октябрь~2011}
  {Разработчик}{}
  {Элитмастер}
  {Екатеринбург, Россия}
{
\begin{itemize}
  \item Разработка CMS, основанной на фрэймворке Yii;
  \item Разработка корпоративной системы {1litr};
  \item Разработка и поддержка большого числа веб-сайтов
\end{itemize}
}


\section{Языки}
\cvlanguage{Русский}{Родной}{}
\cvlanguage{English}{Intermediate}{}

\section{Навыки}
\subsection{Языки программирования и средства разработки}
\cvcomputer{Ruby}{Ruby~on~Rails, Sinatra}
{JavaScript}{JQuery}
\cvcomputer
{Имел опыт работы} {Go, Java, Python, C\#}
{}{}
\subsection{Базы Данных}
\cvcomputer{RDBMS}{PostgreSQL}
           {NoSQL}{Redis, CouchDB}

\subsection{Прочее}
\cvcomputer{Поиск} {Elasticsearch, Sphinx}
           {SCM}{Git}

\section{Социальные сети}
\cvitem{GitHub}{\href{https://github.com/unkmas}{github.com/unkmas}}
\cvitem{VK}{\href{http://vk.com/unkmas}{vk.com/unkmas}}{}{}

\section{Основные достижения}

\cvitem{2013}{
  Победитель конкурса "Участник молодёжного научно-инновационного конкурса"("У.М.Н.И.К.")
}

\cvitem{2013}{
  Участник конференции "АИСТ 2013".
}

\cvitem{2013}{
  Участник международной конференции "Современные проблемы математики".
}

\cvitem{2012}{
  Победитель конкурса "Стипендия от СКБ Контур".
}

\cvitem{2012}{
  Участник RuSSIR 2012.
}

\cvitem{2009}{
  Призёр научно-исследовательской выставки УГТУ--УПИ.
}
\cvitem{2009}{
  Золотой и серебряный сертификаты ректора УГТУ-УПИ.
}

\nocite{*}
\bibliographystyle{unsrt}
%\bibliography{unkmas}

\end{document}
